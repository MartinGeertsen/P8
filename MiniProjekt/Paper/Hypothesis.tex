\section{Hypothesis}
\label{Hypothesis}
When conducting an experiment it is important to specify what exactly we want to measure. As it has already been mentioned, the focus of this study is the perception of the surface of the five dices in comparison with each other. As such, the H0 hypothesis can be formulated as:

\begin{quote}
	\textit{The surface of the dice are perceived equally for each die.} 
\end{quote}

\noindent
Through hypothesis testing, it will take a p-value greater than 0.05 to accept this assumption. If the P-value is lesser than 0.05 there will be strong evidence against the H0 hypothesis, which means that the dices would be perceived differently.

%Null-hypothesis
%Alternate Hypothesis
%Since the surfaces on each dice is different it is believed that there would be different levels of agreement on the attributes. The hypothesis is that the dice that would be ranked as the one with the highest quality will also differ from the other dices in the level of agreement at some attributes. If this is true it would be possible to conclude that the given attribute have a greater influence on the perceived quality