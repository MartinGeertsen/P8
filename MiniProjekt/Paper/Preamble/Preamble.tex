\documentclass[11pt]{article}
% Kommandoerne kan benyttes overalt i rapporten, og synkroniseres således overalt hver gang de opdateres her.

% OBS: Kommandokald som efterfølges af et mellemrum, skal afsluttes med "\" ala; "\groupname\ er en gruppe fra \studyname".

\newcommand{\groupname}{Group 881}
\newcommand{\institutionname}{Electronic Systems}
\newcommand{\adress}{Fredrik Bajers vej 7B}
\newcommand{\city}{9220 Aalborg}
\newcommand{\universityname}{Aalborg University}
\newcommand{\studyname}{Engineering Psychology}
\newcommand{\groupemail}{18g881@es.aau.dk}
\newcommand{\semestername}{Image Processing and Computer Vision}
\newcommand{\semester}{$6^{th}$}
\newcommand{\projectname}{Replicating algorithm}
\newcommand{\projectnameextension}{An Adaptive Algorithm for Grey Image Edge Detection Based on Grey
Correlation Analysis} % Eventually leave empty
\newcommand{\projectnameextended}{\projectname \projectnameextension} %This one is defined by the others.

\newcommand{\supervisor}{???}
\newcommand{\groupmemberI}{Alexander Flyvholm Povlsen}
\newcommand{\groupmemberII}{karolis Mikuta Nielsen}
\newcommand{\groupmemberIII}{kasper Horslev Ravnkilde}
\newcommand{\groupmemberIV}{Martin Geertsen}
\newcommand{\groupmembers}{\groupmemberI, \groupmemberII, \groupmemberIII\ \& \groupmemberIV}

\newcommand{\begindate}{March $05^{th}$}
\newcommand{\finishdate}{April $8^{th}$}
\newcommand{\beginyear}{} %Leave empty if same as \endyear
\newcommand{\finishyear}{2018} 
\newcommand{\projectperiod}{\begindate\beginyear\ to \finishdate\ \finishyear} %This one is defined by the others.
\newcommand{\numberofpages}{3}
\newcommand{\numberofappendix}{0}

%Text
\usepackage[utf8]{inputenc} % Encodes text as UTF8
\usepackage[english]{babel} % Language
\usepackage[T1]{fontenc} % More font options
\usepackage{titling} % Enables reuse of title with \thetitle
\newenvironment{quoteemph}{\noindent\begin{quote}\itshape\bfseries}{\end{quote}} % Emphasised quote with \begin{quoteemph}...\end{quoteemph}

%Math
\usepackage{amsmath,amsfonts,amsthm,amssymb}

%Graphics
\usepackage[pdftex]{graphicx}
\graphicspath{{Graphics/}}
\usepackage{pdfpages}
\usepackage{tcolorbox} % Makes colored text boxes with \begin{tcolorbox}...\end{tcolorbox}
\usepackage{float}
\usepackage[a4paper,pdftex]{geometry}
\usepackage{xcolor} % Define new colors with \definecolor.
%Color palette that looks great
\definecolor{xRed}{HTML}{B51E0E}		%[0.71 0.12 0.06]
\definecolor{xGreen}{HTML}{3B8333}	%[0.23 0.51 0.20]
\definecolor{xBlue}{HTML}{074E82}	%[0.03 0.31 0.51]
\definecolor{xBrown}{HTML}{9C5C19}	%[0.61 0.36 0.10]
\definecolor{xYellow}{HTML}{F7B538}	%[0.97 0.71 0.22]
\definecolor{xOrange}{HTML}{EC6D00}	%[0.93 0.43 0.00]
\definecolor{xCyan}{HTML}{0094AC}	%[0.00 0.58 0.68]
\definecolor{xPurple}{HTML}{8711A1}	%[0.53 0.07 0.64]
\definecolor{xPink}{HTML}{D30580}	%[0.83 0.02 0.51]

%Grey scale (Linear)
\definecolor{xBlack}{HTML}{000000}	%[0.00 0.00 0.00]
\definecolor{xGray75}{HTML}{404040}	%[0.25 0.25 0.25]
\definecolor{xGray50}{HTML}{7F7F7F}	%[0.50 0.50 0.50]
\definecolor{xGray25}{HTML}{BFBFBF}	%[0.75 0.75 0.75]
\definecolor{xGray10}{HTML}{F2F2F2}	%[0.90 0.90 0.90]
\definecolor{xWhite}{HTML}{FFFFFF}	%[1.00 1.00 1.00]

% Example for text: \textcolor{xCyan}{cyan}
% Is often better in BF: \textcolor{xCyan}{\textbf{cyan}} % Set of pre-defined colors
%MATLAB code
\definecolor{mygreen}{RGB}{28,172,0} % color values Red, Green, Blue
\definecolor{mylilas}{RGB}{170,55,241}
\usepackage{listings}
\lstset{language=Matlab,%
    %basicstyle=\color{red},
    breaklines=true,%
    morekeywords={matlab2tikz},
    keywordstyle=\color{blue},%
    morekeywords=[2]{1}, keywordstyle=[2]{\color{black}},
    identifierstyle=\color{black},%
    stringstyle=\color{mylilas},
    commentstyle=\color{mygreen},%
    showstringspaces=false,%without this there will be a symbol in the places where there is a space
    numbers=left,%
    numberstyle={\tiny \color{black}},% size of the numbers
    numbersep=9pt, % this defines how far the numbers are from the text
    emph=[1]{for,end,break},emphstyle=[1]\color{red}, %some words to emphasise
    %emph=[2]{word1,word2}, emphstyle=[2]{style},    
}

\usepackage[margin=10pt,font=small,labelfont=bf]{caption} % Prettifies captions
% Referencing
\usepackage{hyperref} % Clickable links in the PDF
\newcommand*{\fullref}[1]{\hyperref[{#1}]{\autoref*{#1} (\textit{\nameref*{#1}})}} % Makro: \fullref{LABEL_HERE}. Reference a section + it's name

%Bibliography
% Setting up BibLaTeX
\usepackage[
backend=biber,
sorting=nty,
style=authoryear,
citestyle=authoryear,
maxnames=2,
date=iso8601, %Date format YYYY/MM/DD
urldate=iso8601 %Date format YYYY/MM/DD
]{biblatex}

\addbibresource{Bibliography/Bibliography.bib}
\ExecuteBibliographyOptions{firstinits=true,maxnames=2}

\setlength{\bibitemsep}{12pt}
\setlength{\bibhang}{0.2cm}

\AtBeginBibliography{
  \renewcommand*{\multinamedelim}{\addsemicolon\space}
  \renewcommand*{\finalnamedelim}{\addsemicolon\space}
}

\usepackage{xpatch}
\xpretobibmacro{author}{\mkbibbold\bgroup}{}{}
\xapptobibmacro{author}{\egroup}{}{}
\xpretobibmacro{bbx:editor}{\mkbibbold\bgroup}{}{}
\xapptobibmacro{bbx:editor}{\egroup}{}{}

\renewcommand*{\labelnamepunct}{\mkbibbold{\addcolon\space}}

%\nocite{*} %Typeset all entries in .bib file, even if not used in document

%Commentary
\usepackage{lipsum} % Dummy text \lipsum[..]
\usepackage[footnote,marginclue,draft,danish,silent,nomargin]{fixme}
\newcommand{\fxsource}[1]{\fxnote{Source? #1}} % Asking for a source
\newcommand{\fxwrite}[1]{\lipsum[1]\fxnote{Write #1}} % Asking to be written
